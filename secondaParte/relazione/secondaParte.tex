\documentclass[13pt]{article}
\usepackage[utf8]{inputenc}
\title{Programmazione Concorrente e Distribuita\\Seconda Parte}
\author{Suierica Bogdan Ionut 1008089}
\usepackage{graphicx}
\begin{document}
\maketitle

\begin{figure}[h]
\centering
\includegraphics[width=0.7\linewidth]{../../../../../Dropbox/touslesmemes/RelazioneExtra/img/unipd}
\label{fig:unipd}
\end{figure}

\newpage
\section{Cambiamenti e aggiunte}
In questa sezione verranno descritti i cambiamenti apportati alla versione precedente del programma in modo tale che la logica dell'algoritmo di risoluzione del puzzle sia concorrente. I cambiamenti effettuati sono minimi, in particolare è stato cambiato il metodo \textit{solve()} dell'interfaccia \textit{AlgorithmSolver} e l'implementazione di tale metodo.
\begin{figure}[h]
\centering
\includegraphics[width=0.9\linewidth]{../../../uml/secondaParte}
\label{fig:secondaParte}
\end{figure}

La classe PuzzleToSolve implementa l'interfaccia SolverAlgorithm. Oltre ad avere il compito di salvarsi il puzzle disordinato e restituirlo ordinato attraverso il metodo realizzato "solve()", fornito dal interfaccia SolverAlgorithm, deve costruire l'oggetto "Puzzle" da cui verranno copiati tutti gli elementi del puzzle.
\\
Essendo disponibile l'oggetto "Puzzle" è stato tolto il parametro alla funzione solve() dell'interfaccia \textit{SolverAlgorithm} che viene comunque utilizzata per permettere una futura estensibilità del codice. È quindi possibile inserire un differente algoritmo di risoluzione, presentando quindi alternative diverse allo sviluppatore e/o utilizzatore essendo necessaria la sola implementazione dell'interfaccia e del suo metodo \textit{solve()}.

\newpage
\section{Algoritmo di risoluzione}
L'algoritmo di risoluzione è di tipo concorrente come richiesto dalla specifica relativa alla seconda parte del progetto.
\\
Ci sono più attività concorrenti che ricostruiscono il puzzle. I vari thread coinvolti hanno un carico di lavoro uniforme e non tengono occupata la CPU inutilmente. Non ci sono problemi di interferenza e di deadlock.
\\
Le strutture dati utilizzate non cambiano rispetto alla prima parte del progetto e vengono utilizzate sempre 2 \textit{ArrayList} di oggetti \textit{Tile}: uno per il salvataggio degli elementi del puzzle disordinati da file e uno per il salvataggio degli elementi del puzzle ordinati. Al momento del salvataggio degli elementi disordinati da file viene calcolato e memorizzato il numero delle righe e delle colonne in modo da poter visualizzare l'ArrayList come un array a 2 dimensioni. 
\\
La logica della risoluzione del puzzle è molto simile alla logica della prima parte del progetto:
\begin{itemize}
	\item \textbf{Ordino la colonna più a sinistra:}
	Attraverso un ciclo for che scorre le righe del puzzle inizio con la ricerca del primo elemento del puzzle che avrà id\_nord e id\_ovest uguali a "0". Gli altri elementi li trovo cercando l'elemento con id\_ovest sempre uguale a "0" i id\_nord uguale al id dell'elemento precedentemente trovato cioè il primo elemento della riga precedente.
	\item \textbf{Ordino il resto del puzzle}
	Una volta completato il passo precedente, il main thread potrà lanciare l'avvio dei thread che si occupano del ordinamento del resto del puzzle partendo dalla seconda colonna in quanto la prima è stata già ordinata. Verranno avviati tanti task quante saranno il numero delle righe che compongono il puzzle. Per ogni riga, partendo dalla seconda colonna, si confronterà l'id\_ovest del elemento in questione con id dell'elemento a sinistra, nel primo caso con l'elemento della prima colonna. Solo quando tutti i task avranno finito l'ordinamento il main thread potrà continuare, effettuando tutte le operazioni, e terminare la sua esecuzione. 
\end{itemize}
 
\newpage
\section{Gestione dei Thread}
Di seguito verranno descritte le conseguenze che possono presentarsi con l'avvio dei thread necessari per la risoluzione del puzzle.
\subsection{Numero Thread}
Il numero di thread cambia in base al numero di righe del puzzle, pertanto il numero minimo di thread attivi può esser uno. 
\subsection{Interferenze, Deadlock, Attesa attiva}
\begin{itemize}
	\item \textbf{Interferenze}
	Questo fenomeno non può mai presentarsi sull'oggetto condiviso in quanto ogni accesso da parte dei vari thread in esecuzione è effettuato in sola lettura;
	\item \textbf{Deadlock}
	Questo fenomeno non può mai presentarsi in quanto il codice è stato progettato senza lock;
	\item \textbf{Attesa attiva}
	Questo fenomeno non può mai presentarsi in quanto tutte le iterazione hanno condizioni di uscita date dalle precondizione del programma.
\end{itemize}
È stato scelto di non usare nessun costrutto di concorrenza per avere una soluzione quanto semplice.

\section{Compilazione}
Dalla root principale è possibile avviare il commando per la compilazione di tutti i file attraverso l'istruzione \textbf{make}. Se si vuole lanciare il programma si deve utilizzare il seguenti script bash:
\begin{itemize}
	\item \textit{sh PuzzleSolver.sh [input\_file][output\_file]} : questo script riceve in input 2 parametri, file di input e file di output.
\end{itemize}
Prima di procedere con il lancio degli script bash si deve aver compilato.

 
\end{document}